\usepackage[utf8]{inputenc}
\usepackage[spanish,es-lcroman,es-notilde]{babel} % 

% Márgenes
\usepackage[scale=0.75, top=2cm, bottom=3cm]{geometry}

% Lastpage para el footer (x paginas de y)
% Fontawesome para iconos en la cabecera
% Xpatch para cambiar el formato del titulo
\usepackage{lastpage,fontawesome,xpatch}

% Sacar el 'subject' del PDF (era 'Resume of...')
\AtEndPreamble{\hypersetup{pdfsubject=}}

% Espaciar más los items
\let\OLDitemize\itemize
\renewcommand\itemize{\OLDitemize\addtolength{\itemsep}{2pt}}

% Comando de mes en español, para el footer
\newcommand{\MONTH}{%
  \ifcase\the\month
  \or Enero 
  \or Febrero 
  \or Marzo 
  \or Abril 
  \or Mayo 
  \or Junio 
  \or Julio 
  \or Agosto 
  \or Septiembre 
  \or Octubre 
  \or Noviembre 
  \or Diciembre 
  \fi
}

% Cambiar formato de título (en vez de 'Autor | Curriculum' ahora es 'Autor \n Curriculum')
\makeatletter
\xpatchcmd{\makehead}{\titlestyle{~|~\@title}}{\par\vskip1ex\titlestyle{\@title}}{}{}
\makeatother

% Achicar el subtítulo ('Curriculum Vitae')
\renewcommand*{\titlefont}{\fontsize{21}{25}\mdseries\upshape} 

% Header y Footer
\makeatletter
\fancyhead[R]{\color{gray}\@firstname{}~\@familyname{}}
\fancyfoot[R]{\color{gray}\textit{\MONTH \the\year \\ \thepage/\pageref*{LastPage}}}
\makeatother

% Solo footer para la primera pagina
\fancypagestyle{onlyfooter}{
\fancyhf{}
\fancyfoot[R]{\color{gray}\textit{\MONTH \the\year \\ \thepage/\pageref*{LastPage}}}
}
