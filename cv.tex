\documentclass[11pt,a4paper]{moderncv}
\usepackage{pdfpages}
\usepackage{import} \import{./}{setup.tex}

\name{Federico}{del Mazo}
\title{
    \spa{Curriculum Vitae}
    \en{Résumé}
}

\phone[mobile]{+54 911 6110 1997}
\phone[fixed]{011 4656 4494}
\email{fdelmazo@fi.uba.ar}
\extrainfo{\href{https://www.linkedin.com/in/fdelmazo/}{\faLinkedin \vspace{0.4mm} FdelMazo} • \href{https://www.github.com/fdelmazo/}{\faGithub \vspace{0.4mm} FdelMazo}}
\quote{
    \spa{21 años, estudiante de Ingeniería en Informática, desarrollador en Raico S.A.}
    \en{21 years old, Computer Engineering Student, developer at Raico S.A.}
}

\begin{document}


\spa{
    \begin{textblock*}{1.51cm}(19cm,0.2cm) 
        \begin{shaded*}
        \centering
            \href{https://fdelmazo.github.io/CV/cv-en.pdf}{SPA}
        \end{shaded*}
    \end{textblock*}
}

\en{
    \begin{textblock*}{1.51cm}(19cm,0.2cm) 
        \begin{shaded*}
        \centering
            \href{https://fdelmazo.github.io/CV/cv-es.pdf}{ENG}
        \end{shaded*}
    \end{textblock*}
}

\begin{picture}(0,0)
\put(-30,-110){\includegraphics[scale=0.20]{fotoCV}}
\end{picture}

\thispagestyle{onlyfooter}

\makecvtitle
\addtolength{\parskip}{6pt}

\spa{Estudiante en búsqueda de seguir desarrollándose en la industria y enriquecer sus conocimientos.}    
\en{Student eager to delve deeper into the IT industry and looking forward to improve his knowledge and skills.}

\section{\spa{Experiencia Laboral}\en{Work Experience}}

\cventry
    {\spa{Abril 2018--Actualidad}\en{April 2018--Now}}
    {\spa{Desarrollador de software}\en{Full Stack developer}}
    {Raico S.A.}{}{}{\url{https://www.raiconet.com/}\\
    \spa{Mantenimiento y desarrollo constante de la aplicación mobile y de la aplicación web Raiconet, de Raico S.A, empresa altamente especializada en servicios de transporte internacional, tanto de importación como de exportación. El sistema es usado diariamente por más de 50 operadores de la empresa y está diseñado para manejar la logística de empaque y facturación a los clientes.\\
     Desarrollo de Exporta Simple, una plataforma web integrada con el Ministerio de Producción y Trabajo de Argentina para facilitar a los pequeños productores las operaciones de exportación.}
    \en{Development and support of the web application and the mobile app of Raico S.A, a company specialized in international transport services. The system is used daily by 50+ people in the team for the logistics and billing of packages.\\
      Development of \textit{Exporta Simple}, a web platform which helps small producers with export operations, in integration with the Argentine Ministry of Production.}
    }
\cventry
    {\spa{Agosto 2017--Actualidad}\en{August 2017--Now}}
    {\spa{Colaborador - Algoritmos y Programación II - Curso Wachenchauzer}\en{Teaching assistant - Algorithms and Programming II}}
    {Universidad de Buenos Aires, Facultad de Ingeniería}{}{}{\url{https://algoritmos-rw.github.io/algo2/}\\
    \spa{Clases prácticas, talleres, preparación y corrección de exámenes. \\
    Temas cubiertos: C, manejo de memoria, complejidad algorítmica, tipo de datos abstractos, estructuras de datos (listas enlazadas, tablas de hash, árboles, colas de prioridad), teoría de grafos (árboles de tendidos mínimos, recorridos) y heurísticas de programación (programación dinámica y técnicas greedy).}
    \en{Classes, workshops, making and grading of exams.\\
    Covered topics: C, memory management, algorithmic complexity, abstract data types, data structures (linked lists, hash tables, trees, heap queues), graph theory (minimum spanning tree, traversal) and programming heuristics (dynamic programming, greedy techniques).}}

\section{\spa{Educación}\en{Education}}

\cventry
    {\spa{2015--Actualidad}\en{2015--Now}}
    {\spa{Estudiante de Ingeniería en Informática}\en{Computer Engineering student}}
    {Universidad de Buenos Aires, Facultad de Ingeniería}{}{}{}

\cventry
    {2009--2014}
    {\spa{Bachiller Bilingüe en Economía y Administración}\en{Bilingual bachelors' degree in economics and business administration}}
    {Colegio Ward}{}{}{\textit{\spa{Promedio general}\en{Grade Point Average} 8.29}}

\cventry
    {2012--2013}
    {International General Certificate of Secondary Education (IGCSE)}
    {University of Cambridge}{}{}{\textit{Passed with Merit}}
%     Environmental Management B
%     Geography B
%     First Language Spanish D
%     Economics B
%     Mathematics B
%     Business Studies C
%     First language english D

\cventry
    {2011}
    {First Certificate in English}
    {University of Cambridge}{}{}{\textit{Grade C}}
% Cambridge ESOL Level 1 Certificate in ESOL International

\section{\spa{Conocimientos específicos}\en{Specific skills}}

\begin{itemize}
\item \textbf{\spa{Lenguajes de programación}\en{Programming languages}:} C, Python, Java, JavaScript.

\item \textbf{\spa{Paradigmas de programación y técnicas de diseño de algoritmos}\en{Programming paradigms and algorithm design techniques}:} 
    \spa{Programación procedural, programación orientada a objetos, programación dinámica, división y conquista, metodologías greedy}
    \en{Procedural programming, object-oriented programming, dynamic programming, divide and conquer, greedy algorithms}

\item \textbf{\spa{Otros tópicos}\en{Other topics}:}
    \spa{Análisis de datos, complejidad computacional, machine learning, MapReduce, teoría de grafos, compresión de datos, PageRank, hashing}
    \en{Data analysis, computational complexity, machine learning, MapReduce, graph theory, data compression, PageRank, hashing}

\item \textbf{\spa{Otros lenguajes}\en{Other languages}:} SQL, TeX.

\item \textbf{\spa{Frameworks y otros}\en{Frameworks and miscellaneous}:} Groovy, Grails, Pandas, PySpark.

\end{itemize}

\IfFileExists{./notas.pdf}{\includepdf[pages=-]{notas.pdf}}{}
\end{document}

